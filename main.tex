\documentclass[12pt,a4paper]{article}

\usepackage[spanish]{babel}
\usepackage[utf8]{inputenc}
\usepackage{geometry}
\usepackage{listings}
\usepackage{xcolor}
\usepackage{longtable}
\usepackage{hyperref}
\hypersetup{
    colorlinks=true,
    linkcolor=black,
    urlcolor=blue,
    citecolor=black
}

\geometry{margin=2.5cm}

\lstset{
    language=bash,
    backgroundcolor=\color{gray!10},
    basicstyle=\ttfamily\small,
    keywordstyle=\color{blue},
    commentstyle=\color{green!60!black},
    stringstyle=\color{red},
    breaklines=true,
    frame=single
}

\title{\textbf{Guía Maestra de Administración y Scripting en Linux}}
\author{Bryan Mamani}
\date{\today}

\begin{document}

\maketitle
\tableofcontents
\newpage

%====================================================
\section{Comandos Básicos}

\subsection{Gestión de Directorios}

\textbf{ls}

El comando \texttt{ls} permite listar el contenido de un directorio. Puede mostrar archivos, carpetas, permisos, tamaños y fechas de modificación dependiendo de las opciones utilizadas. Es uno de los comandos más usados para inspeccionar la estructura del sistema de archivos.

\begin{lstlisting}
ls
ls -l /home
\end{lstlisting}

\textbf{cd}

El comando \texttt{cd} (change directory) se utiliza para cambiar el directorio de trabajo actual. Permite moverse entre rutas absolutas y relativas dentro del sistema.

\begin{lstlisting}
cd Documentos
cd ..
\end{lstlisting}

\textbf{mkdir}

El comando \texttt{mkdir} crea nuevos directorios. Puede generar uno o varios directorios y también crear estructuras completas usando la opción \texttt{-p}.

\begin{lstlisting}
mkdir proyectos
mkdir -p curso/linux/bash
\end{lstlisting}

\textbf{pwd}

El comando \texttt{pwd} (print working directory) muestra la ruta absoluta del directorio actual. Es útil para saber exactamente en qué ubicación del sistema estamos trabajando.

\begin{lstlisting}
pwd
pwd -P
\end{lstlisting}

%====================================================
\subsection{Manipulación de Archivos}

\textbf{cp}

El comando \texttt{cp} permite copiar archivos y directorios. Puede trabajar de forma recursiva con la opción \texttt{-r} cuando se trata de carpetas.

\begin{lstlisting}
cp archivo.txt copia.txt
cp -r carpeta respaldo/
\end{lstlisting}

\textbf{mv}

El comando \texttt{mv} mueve o renombra archivos y directorios. Es útil tanto para reorganizar archivos como para cambiar su nombre.

\begin{lstlisting}
mv archivo.txt nuevo.txt
mv archivo.txt /tmp
\end{lstlisting}

\textbf{rm}

El comando \texttt{rm} elimina archivos o directorios. Para eliminar carpetas se debe usar la opción \texttt{-r} (recursivo).

\begin{lstlisting}
rm archivo.txt
rm -r carpeta
\end{lstlisting}

\textbf{touch}

El comando \texttt{touch} crea archivos vacíos o actualiza la fecha de modificación de un archivo existente.

\begin{lstlisting}
touch nuevo.txt
touch archivo{1..3}.txt
\end{lstlisting}

\textbf{cat}

El comando \texttt{cat} muestra el contenido de archivos en pantalla y también puede combinar múltiples archivos.

\begin{lstlisting}
cat archivo.txt
cat archivo1.txt archivo2.txt
\end{lstlisting}

%====================================================
\subsection{Ayuda y Manuales}

\textbf{man}

El comando \texttt{man} muestra el manual oficial de un comando, incluyendo descripción, opciones y ejemplos.

\begin{lstlisting}
man ls
man chmod
\end{lstlisting}

\textbf{help}

El comando \texttt{help} muestra ayuda interna para comandos integrados en Bash.

\begin{lstlisting}
help cd
help echo
\end{lstlisting}

%====================================================
\section{Comandos Avanzados}

\subsection{Permisos y Propietarios}

\textbf{chmod}

El comando \texttt{chmod} modifica los permisos de lectura (r), escritura (w) y ejecución (x) de archivos y directorios, ya sea en formato simbólico o numérico.

\begin{lstlisting}
chmod 755 script.sh
chmod +x script.sh
\end{lstlisting}

\textbf{chown}

El comando \texttt{chown} cambia el propietario y/o grupo de un archivo o directorio.

\begin{lstlisting}
chown usuario archivo.txt
chown usuario:grupo archivo.txt
\end{lstlisting}

%====================================================
\subsection{Filtros y Procesamiento}

\textbf{grep}

Busca patrones de texto dentro de archivos. Es muy usado para filtrar información específica.

\begin{lstlisting}
grep "error" archivo.log
grep -i linux archivo.txt
\end{lstlisting}

\textbf{find}

Busca archivos y directorios según nombre, tipo o fecha.

\begin{lstlisting}
find . -name "*.txt"
find /home -type d
\end{lstlisting}

\textbf{head}

Muestra las primeras líneas de un archivo.

\begin{lstlisting}
head archivo.txt
head -n 5 archivo.txt
\end{lstlisting}

\textbf{tail}

Muestra las últimas líneas y puede monitorear archivos en tiempo real.

\begin{lstlisting}
tail archivo.txt
tail -f log.txt
\end{lstlisting}

\textbf{sort}

Ordena líneas de texto alfabéticamente o numéricamente.

\begin{lstlisting}
sort archivo.txt
sort -r archivo.txt
\end{lstlisting}

\textbf{wc}

Cuenta líneas, palabras y caracteres de un archivo.

\begin{lstlisting}
wc archivo.txt
wc -l archivo.txt
\end{lstlisting}

%====================================================
\section{Programación Shell}

\subsection{Estructura Básica}

Todo script Bash comienza con el \textbf{shebang}, que indica qué intérprete ejecutará el archivo.

\begin{lstlisting}
#!/bin/bash
echo "Hola Mundo"
\end{lstlisting}

\subsection{Variables y Argumentos}

Las variables almacenan datos y los argumentos permiten recibir información desde la línea de comandos.

\begin{lstlisting}
#!/bin/bash
nombre=$1
echo "Hola $nombre"
\end{lstlisting}

\subsection{Estructuras de Control}

\textbf{if-else}
\begin{lstlisting}
if [ $1 -gt 10 ]; then
  echo "Mayor que 10"
else
  echo "Menor o igual a 10"
fi
\end{lstlisting}

\textbf{for}
\begin{lstlisting}
for i in {1..5}
do
  echo $i
done
\end{lstlisting}

\textbf{while}
\begin{lstlisting}
contador=1
while [ $contador -le 5 ]
do
  echo $contador
  contador=$((contador+1))
done
\end{lstlisting}

%====================================================
\section{Tuberías y Redireccionamientos}

En Linux todo funciona como flujo de datos. Los procesos reciben información (stdin), generan salida (stdout) y errores (stderr).

\textbf{|} Conecta la salida de un comando con la entrada de otro.
\begin{lstlisting}
ls | grep ".txt"
\end{lstlisting}

\textbf{>} Redirige la salida sobrescribiendo.
\begin{lstlisting}
ls > lista.txt
\end{lstlisting}

\textbf{>} Añade información sin borrar contenido previo.
\begin{lstlisting}
echo "Hola" >> archivo.txt
\end{lstlisting}

\textbf{2>} Redirige errores.
\begin{lstlisting}
ls directorio_inexistente 2> errores.log
\end{lstlisting}

%====================================================
\section{Anexo de Transparencia IA}

\begin{longtable}{|p{4cm}|p{6cm}|p{4cm}|}
\hline
IA Utilizada & Prompt enviado & Parte generada \\ \hline
ChatGPT & "Para la elaboración de este manual, realicé la investigación de los comandos, su funcionamiento, ejemplos prácticos y el análisis de errores comunes por cuenta propia.   
" & La inteligencia artificial fue utilizada únicamente como apoyo para dar formato en LaTeX y mejorar la redacción en algunas secciones, con el fin de que el documento tenga una presentación más clara y ordenada.  \\ \hline
\end{longtable}

\end{document}